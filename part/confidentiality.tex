\section{Механизмы обеспечения конфиденциальности в СУБД}

4.1 Классификация угроз конфиденциальности СУБД
Причины, виды, основные методы нарушения конфиденциальности. Типы утечки конфи-денциальной информации из СУБД, частичное разглашение. Соотношение защищенности и доступности данных. Получение несанкционированного доступа к конфиденциальной инфор-мации путем логических выводов. Методы противодействия. Особенности применения криптографических методов.

4.2 Средства идентификации и аутентификации
Общие сведения. Совместное применение средств идентификации и аутентификации, встроенных в СУБД и  в ОС. 

4.3 Средства управления доступом
Основные понятия: субъекты и объекты, группы пользователей, привилегии, роли и пред-ставления. Виды привилегий: привилегии безопасности и доступа. Использование ролей и привилегий пользователей. Соотношение прав доступа, определяемых ОС и СУБД. Метки безопасности. Использование представлений для обеспечения конфиденциальности информации в СУБД.

4.4 Обеспечение конфиденциальности путем тиражирования БД 
Формальная модель для обеспечения конфиденциальности БД с помощью тиражирования. Архитектура и политика безопасности в модели SINTRA. 

4.5 Аудит и подотчетность
Подотчетность действий пользователя и аудит связанных с безопасностью событий. Реги-страция действий пользователя. Управление набором регистрируемых событий. Анализ регистрационной информации. 