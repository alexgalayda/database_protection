\section{Механизмы обеспечения конфиденциальности в СУБД}
\subsection{Классификация угроз конфиденциальности СУБД}
\subsubsection{Причины, виды, основные методы нарушения конфиденциальности.}

\subsubsection{Причины, виды, основные методы нарушения конфиденциальности}

Основными причинами нарушения конфиденциальности информации являются:
-     несоблюдение персоналом норм, требований, правил эксплуатации АС;
-     ошибки в проектировании АС и систем защиты АС;
-     ведение противостоящей стороной технической и агентурной разведок.


Источник: http://www.o-salo.narod.ru/page4.htm#\_Toc287060195

Угрозы конфиденциальности информации направлены на несанкционированное перемещение информации от носителя-источника к носителю-получателю (угроза разглашения, утечки, несанкционированного доступа).
Уязвимость информации к преднамеренным действиям злоумышленников или иных заинтересованных лиц является наиболее опасной. Как уже отмечалось выше, различные виды уязвимости обусловлены угрозами разглашения, НСД %%утечки информации.
Уязвимость информации к разглашению – несанкционированному сообщению защищаемой информации лицам, не меющим %права доступа к ней, – определяется, прежде всего, неправильной организацией работы с информацией и е %носителями, а также неосторожными или умышленными действиями людей, допущенных к работе с данной информацией.

Уязвимость информации к НСД – противоправному преднамеренному овладению защищаемой информацией лицом, не имеющим права доступа к ней, – определяется наличием у информации, ее носителя или у самой системы защиты недостатков, которые создают возможность получения информации, ее модификации, уничтожения, блокирования доступа к ней.

Уязвимость информации к утечке – неконтролируемому распространению защищаемой информации за пределы круга лиц, которым эта информация была доверена, – возникает, как и в случае разглашения, из-за неправильной организации работы с информацией, а также в связи с "дырами" в системе защиты информации, неосторожными или умышленными действиями людей, допущенных к работе с защищаемыми сведениями. Утечка может происходить по акустическому, визуально-оптическому, материально-вещественному и электромагнитному каналам. Следовательно, уязвимость информации к утечке определяется слабыми сторонами носителя информации, среды, окружающей информацию, и технических средств, находящихся в непосредственной близости от носителей информации.

Источник: http://213.182.177.142/kafedr/19.Special\%27nih\_informacionnih
\_tehnologii/teor\_inf\_bez\_i\_met\_sashit\_inf3/lec/index9.htm

Также, угрозы делят на умышленные и неумышленные(баги в ПО, сгнили провода и тп), умышленные в свою очередь делят на активные и пассивные:

Активные угрозы имеют целью нарушение нормального функционирования ИТ посредством целенаправленного воздействия на аппаратные, программные и информационные ресурсы.

Раскрытие конфиденциальной информации - это бесконтрольный выход конфиденциальной информации за пределы информационной технологии или круга лиц, которым она была доверена по службе или стала известна в процессе работы.

Умышленные угрозы подразделяются также на следующие виды:

Внутренние - возникают внутри управляемой организации. Они чаще всего сопровождаются социальной напряженностью и тяжелым моральным климатом на экономическом объекте, который провоцирует специалистов выполнять какие-либо правонарушения по отношению к информационным ресурсам

Внешние - Пассивные угрозы направлены на несанкционированное использование информационный ресурсов, не оказывая при этом влияния на функционирование ИТ

Далее приведены некоторые угрозы ИБ:

\textbf{Раскрытие конфиденциальной информации} - это бесконтрольный выход конфиденциальной информации за пределы информационной технологии или круга лиц, которым она была доверена по службе или стала известна в процессе работы.

Раскрытие конфиденциальной информации может быть следствием:
\begin{itemize}
    \item разглашения конфиденциальной информации
    \item утечки информации по различным, главным образом техническим, каналам (по визуально-оптическим, акустическим, электромагнитным и др
    \item несанкционированного доступа к конфиденциальной информации различными способами
\end{itemize}

\textbf{Несанкционированный доступ к информации} выражается в противоправном преднамеренном овладении конфиденциальной информацией лицом, не имеющим права доступа к охраняемым сведениям.

Наиболее распространенными путями несанкционированного доступа к информации являются:
\begin{itemize}
    \item перехват электронных излучений
    \item принудительное электромагнитное облучение (подсветка) линий связи с целью получения паразитной модуляции несущей
    \item применение подслушивающих устройств (закладок)
    \item дистанционное фотографирование
    \item перехват акустических излучений и восстановление текста принтера
    \item чтение остаточной информации в памяти системы после выполнения санкционированных запросов
    \item копирование носителей информации с преодолением мер защиты
    \item маскировка под зарегистрированного пользователя ("маскарад")
    \item использование недостатков языков программирования и операционных систем
\end{itemize}

Перечисленные пути несанкционированного доступа требуют достаточно больших технических знаний и соответствующих аппаратных или программных разработок со стороны взломщика. Например, используются технические каналы утечки - это физические пути от источника конфиденциальной информации к злоумышленнику, посредством которых возможно получение охраняемых сведений. Причиной возникновения каналов утечки являются конструктивные и технологические несовершенства схемных решений либо эксплуатационный износ элементов. Все это позволяет взломщикам создавать действующие на определенных физических принципах преобразователи, образующие присущий этим принципам канал передачи информации - канал несанкционированного доступа.

\textbf{Несанкционированное использование информационных ресурсов}, с одной стороны, является последствиями ее утечки и средством ее компрометации. С другой стороны, оно имеет самостоятельное значение, так как может нанести большой ущерб управляемой системе (вплоть до полного выхода информационной технологии из строя) или ее абонентам.

\textbf{Незаконное использование привилегий.} Любая защищенная технология содержит
средства, используемые в чрезвычайных ситуациях, или средства, которые способны
функционировать с нарушением существующей политики безопасности. Например, на случай
внезапной проверки пользователь должен иметь возможность доступа ко всем наборам
системы. Обычно эти средства используются администраторами, операторами, системными
программистами и другими пользователями, выполняющими специальные функции.
Большинство систем защиты в таких случаях используют наборы привилегий, т. е. для
выполнения определенной функции требуется определенная привилегия. Обычно
пользователи имеют минимальный набор привилегий, администраторы - максимальный.
Наборы привилегий охраняются системой защиты. Несанкционированный (незаконный)
захват привилегий возможен при наличии ошибок в системе защиты, но чаще всего
происходит в процессе управления системой защиты, в частности, при небрежном
пользовании привилегиями.

\textbf{"Взлом системы"} - умышленное проникновение в информационную технологию, когда взломщик не имеет санкционированных параметров для входа. Способы взлома могут быть различными, и при некоторых из них происходит совпадение с ранее описанными угрозами. Например, использование пароля пользователя информационной технологии, который может быть вскрыт, например, путем перебора возможных паролей.

Источник: https://www.intuit.ru/studies/courses/3609/851/lecture/31660


\subsubsection{Типы утечки конфиденциальной информации из СУБД, частичное разглашение}

Утечки информации — неправомерная передача конфиденциальных сведений (материалов, важных для различных компаний или государства, персональных данных граждан), которая может быть умышленной или случайной. Утечка информации возможна по разным причинам.

Умышленные утечки:

Инсайдеры и избыточные права. К этому виду относятся случаи, основной причиной которых стали действия сотрудников, имеющих доступ к секретам легально, в силу своих служебных обязанностей. Все варианты инсайда условно можно разделить на две группы: в одном случае сотрудник, не имея доступа к информации, получает ее незаконно, а в другом он обладает официальным доступом к закрытым данным и умышленно выносит их за пределы компании.

Кража информации (извне). Проникновение в компьютер с помощью вредоносных программ и похищение информации с целью использования в корыстных интересах. На хакерские атаки приходится 15\% от всего объема утечек информации. Вторжение в устройство извне и незаметная установка вредоносных программ позволяют хакерам полностью контролировать систему и получать доступ к закрытым сведениям, вплоть до паролей к банковским счетам и картам. Для этого могут применяться различные программы вроде «троянских коней». Главный атрибут данного вида утечки — активные действия внешних лиц с целью доступа к информации.

Взлом программного обеспечения. Приложения, используемые в рабочем процессе или на личном компьютере сотрудника, часто имеют незакрытые уязвимости, которые можно эксплуатировать и получать тем самым различные небезопасные возможности вроде исполнения произвольного кода или повышения привилегий.

Вредоносные программы (бекдоры, трояны) нацелены на причинение вреда владельцу устройства, позволяют незаметно проникать в систему и затем искажать или полностью удалять информацию, подменять ее другими,  похожими данными.

Кражи носителей. Весьма распространенный вариант утечек, который случается в результате преднамеренного хищения устройств с информацией — ноутбуков, смартфонов, планшетов и других съемных носителей данных в виде флешек, жестких дисков.

Случайные утечки. К этому виду можно отнести инциденты, которые происходят из-за потери носителей данных (флешек, ноутбуков, смартфонов) или ошибочных действий сотрудников организации. Результатом может быть размещение конфиденциальной информации в интернете. Также не последнюю роль играет человеческий фактор, когда сотрудник по недосмотру открывает доступ к закрытым данным всем желающим.

Источник: https://www.anti-malware.ru/threats/leaks

\subsubsection{Соотношение защищенности и
доступности данных}

??? В. Г. Проскурин. Защита в операционных системах 2014.pdf с 7 п4 мб об этом
Суть такая, слишком жесткие политики безопасности могут повлечь нарушение доступности. Пример: запретить поль-лю SYSTEM в WINDOWS доступ к исполняемым файлам - тогда ОС нормально загрузиться не сможет.
%\subsubsection{Получение несанкционированного доступа к конфиденциальной информации путем логических %выводов. Особенности применения
%криптографических методов.}
%Нередко путем логического вывода можно извлечь из базы данных информацию, на получение которой стандартными %средствами у пользователя не хватает привилегий.
%
%Если для реализации контроля доступа используются представления и эти представления допускают модификацию,  %%помощью операций модификации/вставки можно получить информацию о содержимом базовых таблиц, не располагая %прямым доступом к ним.
%
%Основным средством борьбы с подобными угрозами, помимо тщательно проектирования модели данных, является %механизм размножения строк. Суть его в том, что в состав первичного ключа, явно или неявно, включается етка %%безопасности, за счет чего появляется возможность хранить в таблице несколько экземпляров строк с %одинаковыми значениями "содержательных" ключевых полей. Наиболее естественно размножение строк реализуется  %%СУБД, поддерживающих метки безопасности (например, в INGRES/Enhanced Security), однако и стандартными %SQL-средствами можно получить удовлетворительное решение.
%
%рассмотрим базу данных, состоящую из одной таблицы с двумя столбцами - имя пациента и диагноз. %Предполагается, что имя является первичным ключом. Каждая из строк таблицы относится к одному из двух %уровней секретности - высокому (HIGH) и низкому (LOW). Соответственно, и пользователи подразделяются на два %уровня благонадежности, которые мы также будем называть высоким и низким.
%К высокому уровню секретности относятся сведения о пациентах, находящихся под надзором правоохранительных %органов или страдающих специфическими заболеваниями. На низком уровне располагаются данные о прочих %пациентах, а также информация о некоторых "секретных" пациентах с "маскировочным" диагнозом. Части таблицы %могут выглядеть примерно так:
%pic
%Обратим внимание на то, что сведения о пациенте по фамилии Иванов присутствуют на обоих уровнях, но одержат %%разные диагнозы.
%Мы хотим реализовать такую дисциплину доступа, чтобы пользователи с низким уровнем благонадежности могли %манипулировать только данными на своем уровне и не имели возможности сделать какие-либо выводы о рисутствии %%в секретной половине сведений о конкретных пациентах. Пользователи с высоким уровнем лагонадежности должны %%иметь доступ к секретной половине таблицы, а также к информации о прочих пациентах. езинформирующих строк о %%секретных пациентах они видеть не должны:
%pic


%subsection{Классификация угроз конфиденциальности СУБД}

%\subsubsection{Типы утечки конфиденциальной информации из СУБД, частичное разглашение}
%
%Утечки информации — неправомерная передача конфиденциальных сведений (материалов, важных для различных %компаний или государства, персональных данных граждан), которая может быть умышленной или случайной. Утечка %информации возможна по разным причинам.
%
%Умышленные утечки:
%
%Инсайдеры и избыточные права. К этому виду относятся случаи, основной причиной которых стали действия %сотрудников, имеющих доступ к секретам легально, в силу своих служебных обязанностей. Все варианты инсайда %условно можно разделить на две группы: в одном случае сотрудник, не имея доступа к информации, получает ее %незаконно, а в другом он обладает официальным доступом к закрытым данным и умышленно выносит их за пределы %компании.
%
%Кража информации (извне). Проникновение в компьютер с помощью вредоносных программ и похищение информации с %целью использования в корыстных интересах. На хакерские атаки приходится 15\% от всего объема утечек %информации. Вторжение в устройство извне и незаметная установка вредоносных программ позволяют хакерам %полностью контролировать систему и получать доступ к закрытым сведениям, вплоть до паролей к банковским %счетам и картам. Для этого могут применяться различные программы вроде «троянских коней». Главный атрибут %данного вида утечки — активные действия внешних лиц с целью доступа к информации.
%
%Взлом программного обеспечения. Приложения, используемые в рабочем процессе или на личном компьютере %сотрудника, часто имеют незакрытые уязвимости, которые можно эксплуатировать и получать тем самым различные %небезопасные возможности вроде исполнения произвольного кода или повышения привилегий.
%
%Вредоносные программы (бекдоры, трояны) нацелены на причинение вреда владельцу устройства, позволяют %незаметно проникать в систему и затем искажать или полностью удалять информацию, подменять ее другими,  %похожими данными.
%
%Кражи носителей. Весьма распространенный вариант утечек, который случается в результате преднамеренного %хищения устройств с информацией — ноутбуков, смартфонов, планшетов и других съемных носителей данных в виде %флешек, жестких дисков.
%
%Случайные утечки. К этому виду можно отнести инциденты, которые происходят из-за потери носителей данных %(флешек, ноутбуков, смартфонов) или ошибочных действий сотрудников организации. Результатом может быть %размещение конфиденциальной информации в интернете. Также не последнюю роль играет человеческий фактор, %когда сотрудник по недосмотру открывает доступ к закрытым данным всем желающим.
%
%Источник: https://www.anti-malware.ru/threats/leaks

%\subsubsection{Соотношение защищенности и доступности данных}
%???
\subsubsection{Получение несанкционированного доступа к конфиденциальной информации путем логических выводов}
Нередко путем логического вывода можно извлечь из базы данных информацию, на получение которой стандартными средствами у пользователя не хватает привилегий.

Если для реализации контроля доступа используются представления и эти представления допускают модификацию,  %помощью операций модификации/вставки можно получить информацию о содержимом базовых таблиц, не располагая прямым доступом к ним.

Основным средством борьбы с подобными угрозами, помимо тщательно проектирования модели данных, является механизм размножения строк. Суть его в том, что в состав первичного ключа, явно или неявно, включается метка безопасности, за счет чего появляется возможность хранить в таблице несколько экземпляров строк с одинаковыми значениями "содержательных" ключевых полей. Наиболее естественно размножение строк реализуется  СУБД, поддерживающих метки безопасности (например, в INGRES/
Enhanced Security), однако и стандартными SQL-средствами можно получить удовлетворительное решение.

Рассмотрим базу данных, состоящую из одной таблицы с двумя столбцами - имя пациента и диагноз. Предполагается, что имя является первичным ключом. Каждая из строк таблицы относится к одному из двух уровней секретности - высокому (HIGH) и низкому (LOW). Соответственно, и пользователи подразделяются на два уровня благонадежности, которые мы также будем называть высоким и низким.
К высокому уровню секретности относятся сведения о пациентах, находящихся под надзором правоохранительных органов или страдающих специфическими заболеваниями. На низком уровне располагаются данные о прочих пациентах, а также информация о некоторых "секретных" пациентах с "маскировочным" диагнозом. Части таблицы могут выглядеть примерно так:

\begin{figure}[h]
    \centering
    \includegraphics[width=0.8\textwidth]{assets/dick_pic1.png}
\end{figure}

\begin{figure}[h]
    \centering
    \includegraphics[width=0.8\textwidth]{assets/dick_pic2.png}
\end{figure}

Обратим внимание на то, что сведения о пациенте по фамилии Иванов присутствуют на обоих уровнях, но cодержат разные диагнозы.

Мы хотим реализовать такую дисциплину доступа, чтобы пользователи с низким уровнем благонадежности могли манипулировать только данными на своем уровне и не имели возможности сделать какие-либо выводы о присутствии в секретной половине сведений о конкретных пациентах. Пользователи с высоким уровнем благонадежности должны иметь доступ к секретной половине таблицы, а также к информации о прочих пациентах. Дезинформирующих строк о секретных пациентах они видеть не должны:

\begin{figure}[h]
    \centering
    \includegraphics[width=0.8\textwidth]{assets/dick_pic3.png}
\end{figure}

%\subsubsection{Причины, виды, основные методы нарушения конфиденциальности.}
%
%Основными причинами нарушения конфиденциальности информации являются:
%\begin{itemize}
%    \item несоблюдение персоналом норм, требований, правил эксплуатации АС
%    \item ошибки в проектировании АС и систем защиты АС
%    \item ведение противостоящей стороной технической и агентурной разведок
%\end{itemize}
%
%
%Источник: http://www.o-salo.narod.ru/page4.htm#\_Toc287060195
%
%Угрозы конфиденциальности информации направлены на несанкционированное перемещение информации от %носителя-источника к носителю-получателю (угроза разглашения, утечки, несанкционированного доступа).
%
%Уязвимость информации к преднамеренным действиям злоумышленников или иных заинтересованных лиц является %наиболее опасной. Как уже отмечалось выше, различные виды уязвимости обусловлены угрозами разглашения, НСД и %утечки информации.
%
%Уязвимость информации к разглашению – несанкционированному сообщению защищаемой информации лицам, не имеющим %права доступа к ней, – определяется, прежде всего, неправильной организацией работы с информацией и ее %носителями, а также неосторожными или умышленными действиями людей, допущенных к работе с данной %информацией.
%
%Уязвимость информации к НСД – противоправному преднамеренному овладению защищаемой информацией лицом, не %имеющим права доступа к ней, – определяется наличием у информации, ее носителя или у самой системы защиты %недостатков, которые создают возможность получения информации, ее модификации, уничтожения, блокирования %доступа к ней.
%
%Уязвимость информации к утечке – неконтролируемому распространению защищаемой информации за пределы круга %лиц, которым эта информация была доверена, – возникает, как и в случае разглашения, из-за неправильной %организации работы с информацией, а также в связи с "дырами" в системе защиты информации, неосторожными или %умышленными действиями людей, допущенных к работе с защищаемыми сведениями. Утечка может происходить по %акустическому, визуально-оптическому, материально-вещественному и электромагнитному каналам. Следовательно, %уязвимость информации к утечке определяется слабыми сторонами носителя информации, среды, окружающей %информацию, и технических средств, находящихся в непосредственной близости от носителей информации.
%
%Источник: http://213.182.177.142/kafedr/19.Special\%27nih\_informacionnih
%\_tehnologii/teor\_inf\_bez\_i\_met\_sashit\_inf3/lec/index9.htm
%
%Также, угрозы делят на умышленные и неумышленные(баги в ПО, сгнили провода и тп), умышленные в свою очередь %делят на активные и пассивные:
%
%Активные угрозы имеют целью нарушение нормального функционирования ИТ посредством целенаправленного %воздействия на аппаратные, программные и информационные ресурсы.
%
%Раскрытие конфиденциальной информации - это бесконтрольный выход конфиденциальной информации за пределы %информационной технологии или круга лиц, которым она была доверена по службе или стала известна в процессе %работы.
%
%Умышленные угрозы подразделяются также на следующие виды:
%
%Внутренние - возникают внутри управляемой организации. Они чаще всего сопровождаются социальной %напряженностью и тяжелым моральным климатом на экономическом объекте, который провоцирует специалистов %выполнять какие-либо правонарушения по отношению к информационным ресурсам
%
%Внешние - Пассивные угрозы направлены на несанкционированное использование информационный ресурсов, не %оказывая при этом влияния на функционирование ИТ
%
%Далее приведены некоторые угрозы ИБ:
%\textbf{Раскрытие конфиденциальной информации} - это бесконтрольный выход конфиденциальной информации за %пределы информационной технологии или круга лиц, которым она была доверена по службе или стала известна в %процессе работы.
%
%Раскрытие конфиденциальной информации может быть следствием:
%\begin{itemize}
%    \item разглашения конфиденциальной информации
%    \item утечки информации по различным, главным образом техническим, каналам (по визуально-оптическим, %акустическим, электромагнитным и др
%    \item несанкционированного доступа к конфиденциальной информации различными способами
%\end{itemize}
%
%\textbf{Несанкционированный доступ к информации} выражается в противоправном преднамеренном овладении %конфиденциальной информацией лицом, не имеющим права доступа к охраняемым сведениям.
%
%Наиболее распространенными путями несанкционированного доступа к информации являются:
%\begin{itemize}
%    \item перехват электронных излучений
%    \item принудительное электромагнитное облучение (подсветка) линий связи с целью получения паразитной %модуляции несущей
%    \item применение подслушивающих устройств (закладок)
%    \item дистанционное фотографирование
%    \item перехват акустических излучений и восстановление текста принтера
%    \item чтение остаточной информации в памяти системы после выполнения санкционированных запросов
%    \item копирование носителей информации с преодолением мер защиты
%    \item маскировка под зарегистрированного пользователя ("маскарад")
%    \item использование недостатков языков программирования и операционных систем
%\end{itemize}
%
%Перечисленные пути несанкционированного доступа требуют достаточно больших технических знаний и %соответствующих аппаратных или программных разработок со стороны взломщика. Например, используются %технические каналы утечки - это физические пути от источника конфиденциальной информации к злоумышленнику, %посредством которых возможно получение охраняемых сведений. Причиной возникновения каналов утечки являются %конструктивные и технологические несовершенства схемных решений либо эксплуатационный износ элементов. Все %это позволяет взломщикам создавать действующие на определенных физических принципах преобразователи, %образующие присущий этим принципам канал передачи информации - канал несанкционированного доступа.
%\paragraph{Типы утечки конфиденциальной информации из СУБД, частичное разглашение}
%
%Утечки информации — неправомерная передача конфиденциальных сведений (материалов, важных для различных %компаний или государства, персональных данных граждан), которая может быть умышленной или случайной. Утечка %информации возможна по разным причинам.
%
%Умышленные утечки:
%
%Инсайдеры и избыточные права. К этому виду относятся случаи, основной причиной которых стали действия %сотрудников, имеющих доступ к секретам легально, в силу своих служебных обязанностей. Все варианты инсайда %условно можно разделить на две группы: в одном случае сотрудник, не имея доступа к информации, получает ее %незаконно, а в другом он обладает официальным доступом к закрытым данным и умышленно выносит их за пределы %компании.
%
%Кража информации (извне). Проникновение в компьютер с помощью вредоносных программ и похищение информации с %целью использования в корыстных интересах. На хакерские атаки приходится 15\% от всего объема утечек %информации. Вторжение в устройство извне и незаметная установка вредоносных программ позволяют хакерам %полностью контролировать систему и получать доступ к закрытым сведениям, вплоть до паролей к банковским %счетам и картам. Для этого могут применяться различные программы вроде «троянских коней». Главный атрибут %данного вида утечки — активные действия внешних лиц с целью доступа к информации.
%
%Взлом программного обеспечения. Приложения, используемые в рабочем процессе или на личном компьютере %сотрудника, часто имеют незакрытые уязвимости, которые можно эксплуатировать и получать тем самым различные %небезопасные возможности вроде исполнения произвольного кода или повышения привилегий.
%
%Вредоносные программы (бекдоры, трояны) нацелены на причинение вреда владельцу устройства, позволяют %незаметно проникать в систему и затем искажать или полностью удалять информацию, подменять ее другими,  %похожими данными.
%
%Кражи носителей. Весьма распространенный вариант утечек, который случается в результате преднамеренного %хищения устройств с информацией — ноутбуков, смартфонов, планшетов и других съемных носителей данных в виде %флешек, жестких дисков.
%
%Случайные утечки. К этому виду можно отнести инциденты, которые происходят из-за потери носителей данных %(флешек, ноутбуков, смартфонов) или ошибочных действий сотрудников организации. Результатом может быть %размещение конфиденциальной информации в интернете. Также не последнюю роль играет человеческий фактор, %когда сотрудник по недосмотру открывает доступ к закрытым данным всем желающим.
%
%Источник: https://www.anti-malware.ru/threats/leaks

%\subsubsection{Соотношение защищенности и
%доступности данных}
%
%??? В. Г. Проскурин. Защита в операционных системах 2014.pdf с 7 п4 мб об этом
%Суть такая, слишком жесткие политики безопасности могут повлечь нарушение доступности. Пример: запретить %поль-лю SYSTEM в WINDOWS доступ к исполняемым файлам - тогда ОС нормально загрузиться не сможет.
%\subsubsection{Получение несанкционированного доступа к конфиденциальной информации путем логических %выводов. Особенности применения
%криптографических методов.}
%Нередко путем логического вывода можно извлечь из базы данных информацию, на получение которой стандартными %средствами у пользователя не хватает привилегий.
%
%Если для реализации контроля доступа используются представления и эти представления допускают модификацию, с %помощью операций модификации/вставки можно получить информацию о содержимом базовых таблиц, не располагая %прямым доступом к ним.
%
%Основным средством борьбы с подобными угрозами, помимо тщательно проектирования модели данных, является %механизм размножения строк. Суть его в том, что в состав первичного ключа, явно или неявно, включается метка %безопасности, за счет чего появляется возможность хранить в таблице несколько экземпляров строк с %одинаковыми значениями "содержательных" ключевых полей. Наиболее естественно размножение строк реализуется в %СУБД, поддерживающих метки безопасности (например, в INGRES/Enhanced Security), однако и стандартными %SQL-средствами можно получить удовлетворительное решение.
%
%рассмотрим базу данных, состоящую из одной таблицы с двумя столбцами - имя пациента и диагноз. %Предполагается, что имя является первичным ключом. Каждая из строк таблицы относится к одному из двух %уровней секретности - высокому (HIGH) и низкому (LOW). Соответственно, и пользователи подразделяются на два %уровня благонадежности, которые мы также будем называть высоким и низким.
%К высокому уровню секретности относятся сведения о пациентах, находящихся под надзором правоохранительных %органов или страдающих специфическими заболеваниями. На низком уровне располагаются данные о прочих %пациентах, а также информация о некоторых "секретных" пациентах с "маскировочным" диагнозом. Части таблицы %могут выглядеть примерно так:
%pic
%Обратим внимание на то, что сведения о пациенте по фамилии Иванов присутствуют на обоих уровнях, но содержат %разные диагнозы.
%Мы хотим реализовать такую дисциплину доступа, чтобы пользователи с низким уровнем благонадежности могли %манипулировать только данными на своем уровне и не имели возможности сделать какие-либо выводы о присутствии %в секретной половине сведений о конкретных пациентах. Пользователи с высоким уровнем благонадежности должны %иметь доступ к секретной половине таблицы, а также к информации о прочих пациентах. Дезинформирующих строк о %секретных пациентах они видеть не должны:
%pic
%\subsubsection{Методы противодействия. Средства идентификации и аутентификации}
%Начнем с понятий идентификации и аутентификации:
%
%Идентификация объекта – это установление эквивалентности между объектом и его априорным обозначением %(определением, представлением, образом, комплексом характеристик). Другими словами, идентификация это %выделение обьекта и установление для него отличительного набора параметров или характеристик. Например для %идентификации пользо- вателя это наличие идентифицирующей его информации, на основе которой можно однозначно %выделить его среди остальных: никнейм (псевдоним), элек- тронная почта, ФИО, СНИЛС и т.п.
%
%Аутентификация – это процедура проверки подлинности входящего в систему обьекта(пользователя), предьявившего %свой идентификатор. При проведении аутентификации прове- рющая сторона удостоверяется в подлинности %проверяемой стороны, однако, проверяемая сторона также учавствует в процессе обмена информацией.
%
%Аутентификация и идентификация являются взаимосвязанными процессами для определения и проверки подлинности %пользователей системы. Именно от них зависит дальнейшее решение системы о том, какие действия следует %предпринимать для предоставления ресурсов пользователю. Следует уточнить, что решение о предоставлении %доступа зависит от используемой информационной системы.
%
%Также стоит ввести понятие строгой аутентификации. Идея строгой аутентификации заключается в том, что %проверяемая сторона доказывает свою подлинность проверяющей стороне на основе какого- либо секрета, который %предварительно был размещен на обеих сторонах.Совместное применение средств идентификации и аутентификации,
%встроенных в СУБД и в ОС.
%
%Основополагающим является тот факт, что проверяемая сторона не передает свой секрет, аутентификация %обеспечивается ответами доказывающей стороны на сгенерированные вопросы проверяющей стороны.
%В соответствии с рекомендациями стандарта Х.509 различают процедуры строгой аутентификации следующих типов:
%– односторонняя аутентификация – двусторонняя аутентификация – трехсторонняя аутентификация
%Односторонняя аутентификация предусматривает обмен информацией только в одном направлении.
%Двусторонняя аутентификация, по сравнению с односторонней, содержит дополнительный ответ проверяющей стороны %доказывающей проверяемой стороне, что передача информации будет осуществляться именно с тем партнером, %которому были предназначены аутентификационные данные.
%Трехсторонняя аутентификация содержит дополнительную передачу данных от доказывающей стороны проверяющей.
%
%Ещё до появления компьютеров использовались различные отличительные черты субъекта, его характеристики. %Сейчас использование той или иной характеристики в системе зависит от требуемой надёжности, защищённости и %стоимости внедрения. Выделяют три фактора аутентификации:
%
%Фактор знания: что-то, что мы знаем — пароль. Это тайные сведения, которыми должен обладать только %авторизованный субъект. Паролем может быть речевое слово, текстовое слово, комбинация для замка или личный %идентификационный номер (PIN). Парольный механизм может быть довольно легко реализован и имеет низкую %стоимость. Однако он имеет существенные недостатки: сохранить пароль в тайне зачастую бывает сложно, %злоумышленники постоянно придумывают новые способы кражи, взлома и подбора пароля (см. бандитский %криптоанализ, метод грубой силы). Это делает парольный механизм слабозащищённым. Многие секретные вопросы, %такие как «Где вы родились?», элементарные примеры фактора знаний, потому что они могут быть известны %широкой группой людей или быть исследованы.
%
%Фактор владения: что-то, что мы имеем — устройство аутентификации. Здесь важно обстоятельство обладания %субъектом каким-то неповторимым предметом. Это может быть личная печать, ключ от замка, для компьютера это %файл данных, содержащих характеристику. Характеристика часто встраивается в особое устройство %аутентификации, например пластиковую карту, смарт-карту. Для злоумышленника заполучить такое устройство %более сложно, чем взломать пароль, а субъект может сразу же сообщить в случае кражи устройства. Это делает %данный метод более защищённым, чем парольный механизм, однако стоимость такой системы более высокая.
%
%Фактор свойства: что-то, что является частью нас — биометрика. Характеристикой является физическая %особенность субъекта. Это может быть портрет, отпечаток пальца или ладони, голос или особенность глаза. С %точки зрения субъекта, данный способ является наиболее простым: не надо ни запоминать пароль, ни переносить %с собой устройство аутентификации. Однако биометрическая система должна обладать высокой чувствительностью, %чтобы подтверждать авторизованного пользователя, но отвергать злоумышленника со схожими биометрическими %параметрами. Также стоимость такой системы довольно велика. Но, несмотря на свои недостатки, биометрика %остается довольно перспективным фактором.
%
%Источник: https://vk.com/doc56814635\_484903425
%?hash=ed44eb7dc39701c4d1&dl
%=73bf7e7d0e55851484

%\subsubsection{Совместное применение средств идентификации и аутентификации, встроенных в СУБД и в ОС.}
%СУБД Oracle предоставляет следующие способы аутентификации
%пользователей:
%\begin{enumerate}
%    \item Аутентификация средствами операционной системы. Некоторые ОС
%позволяют СУБД Oracle использовать информацию о пользователях, которыми
%управляет собственно ОС. В этом случае пользователь компьютера имеет доступ к
%ресурсам БД без дополнительного указания имени и пароля – используются его
%сетевые учетные данные. Данный вид аутентификации считается небезопасным и
%используется, в основном, для аутентификации администратора СУБД.
%    \item Аутентификация при помощи сетевых сервисов. Данный вид аутентификации
%обеспечивает опция сервера Oracle Advanced Security. Она предоставляет
%возможность SSL-аутентификации, а так же аутентификацию с помощью служб
%третьих сторон, в роли которых могут выступать Kerberos, PKI, RADIUS или служба
%LDAP-каталога.
%    \item Аутентификация в многоуровневых приложениях. Приведенные выше
%методы аутентификации также могут быть применены и в многоуровневых
%приложениях. Как правило, для доступа к приложениям из сети Интернет используется аутентификация по имени и %паролю (в том числе с использованием протокола
%RADIUS), либо по протоколу SSL. Прочие методы используются для работы
%пользователей в локальной сети.
%\end{enumerate}
%
%Источник : http://www.iso27000.ru/chitalnyi-zai/zaschita-personalnyh-dannyh/obespechenie-zaschity-personalny%h-dannyh-v-subd-oracle

\subsubsection{Методы противодействия. Особенности применения криптографических методов}
В целях обеспечения конфиденциальности информации используются следующие криптографические примитивы:
\begin{itemize}
    \item Симметричные криптосистемы
    \item Асимметричные криптосистемы
\end{itemize}

В симметричных криптосистемах для зашифрования и расшифрования информации используется один и тот же общий секретный ключ, которым взаимодействующие стороны предварительно обмениваются по некоторому защищённому каналу. В качестве примеров симметричных криптосистем можно привести международные стандарты DES и пришедший ему на смену AES.

Асимметричные криптосистемы характерны тем, что в них используются различные ключи для зашифрования и расшифрования информации. Ключ для зашифрования (открытый ключ) можно сделать общедоступным, с тем чтобы любой желающий мог зашифровать сообщение для некоторого получателя. Получатель же, являясь единственным обладателем ключа для расшифрования (секретный ключ), будет единственным, кто сможет расшифровать зашифрованные для него сообщения. Примеры асимметричных криптосистем – RSA и схема Эль-Гамаля.

Симметричные и асимметричные криптосистемы, а также различные их комбинации используются в АС прежде всего для шифрования данных на различных носителях и для шифрования трафика.
\subsection{Средства идентификации и аутентификации}

\subsubsection{Общие сведения}
Начнем с понятий идентификации и аутентификации:

Идентификация объекта – это установление эквивалентности между объектом и его априорным обозначением (определением, представлением, образом, комплексом характеристик). Другими словами, идентификация это выделение обьекта и установление для него отличительного набора параметров или характеристик. Например для идентификации пользо- вателя это наличие идентифицирующей его информации, на основе которой можно днозначно %выделить его среди остальных: никнейм (псевдоним), элек- тронная почта, ФИО, СНИЛС и т.п.

Аутентификация – это процедура проверки подлинности входящего в систему обьекта(пользователя), редьявившего %свой идентификатор. При проведении аутентификации прове- рющая сторона удостоверяется в одлинности %проверяемой стороны, однако, проверяемая сторона также учавствует в процессе обмена нформацией.

Аутентификация и идентификация являются взаимосвязанными процессами для определения и проверки подлинности пользователей системы. Именно от них зависит дальнейшее решение системы о том, какие действия следует предпринимать для предоставления ресурсов пользователю. Следует уточнить, что решение о предоставлении доступа зависит от используемой информационной системы.

Также стоит ввести понятие строгой аутентификации. Идея строгой аутентификации заключается в том, что проверяемая сторона доказывает свою подлинность проверяющей стороне на основе какого- либо секрета, который предварительно был размещен на обеих сторонах.Совместное применение средств идентификации и аутентификации,
встроенных в СУБД и в ОС.

Основополагающим является тот факт, что проверяемая сторона не передает свой секрет, аутентификация обеспечивается ответами доказывающей стороны на сгенерированные вопросы проверяющей стороны.
В соответствии с рекомендациями стандарта Х.509 различают процедуры строгой аутентификации следующих типов:
– односторонняя аутентификация – двусторонняя аутентификация – трехсторонняя аутентификация
Односторонняя аутентификация предусматривает обмен информацией только в одном направлении.
Двусторонняя аутентификация, по сравнению с односторонней, содержит дополнительный ответ проверяющей тороны %доказывающей проверяемой стороне, что передача информации будет осуществляться именно с тем артнером, %которому были предназначены аутентификационные данные.
Трехсторонняя аутентификация содержит дополнительную передачу данных от доказывающей стороны проверяющей.

Ещё до появления компьютеров использовались различные отличительные черты субъекта, его характеристики. Сейчас использование той или иной характеристики в системе зависит от требуемой надёжности, защищённости и стоимости внедрения. Выделяют три фактора аутентификации:

Фактор знания: что-то, что мы знаем — пароль. Это тайные сведения, которыми должен обладать только авторизованный субъект. Паролем может быть речевое слово, текстовое слово, комбинация для замка или личный идентификационный номер (PIN). Парольный механизм может быть довольно легко реализован и имеет низкую стоимость. Однако он имеет существенные недостатки: сохранить пароль в тайне зачастую бывает сложно, злоумышленники постоянно придумывают новые способы кражи, взлома и подбора пароля (см. бандитский криптоанализ, метод грубой силы). Это делает парольный механизм слабозащищённым. Многие секретные вопросы, такие как «Где вы родились?», элементарные примеры фактора знаний, потому что они могут быть известны широкой группой людей или быть исследованы.

Фактор владения: что-то, что мы имеем — устройство аутентификации. Здесь важно обстоятельство обладания субъектом каким-то неповторимым предметом. Это может быть личная печать, ключ от замка, для компьютера это файл данных, содержащих характеристику. Характеристика часто встраивается в особое устройство аутентификации, например пластиковую карту, смарт-карту. Для злоумышленника заполучить такое устройство более сложно, чем взломать пароль, а субъект может сразу же сообщить в случае кражи устройства. Это делает данный метод более защищённым, чем парольный механизм, однако стоимость такой системы более высокая.

Фактор свойства: что-то, что является частью нас — биометрия. Характеристикой является физическая особенность субъекта. Это может быть портрет, отпечаток пальца или ладони, голос или особенность глаза. С точки зрения субъекта, данный способ является наиболее простым: не надо ни запоминать пароль, ни переносить с собой устройство аутентификации. Однако биометрическая система должна обладать высокой чувствительностью, чтобы подтверждать авторизованного пользователя, но отвергать злоумышленника со схожими биометрическими параметрами. Также стоимость такой системы довольно велика. Но, несмотря на свои недостатки, биометрика остается довольно перспективным фактором.

Источник: https://vk.com/doc56814635\_484903425
?hash=
ed44eb7dc39701c4d1&dl
=73bf7e7d0e55851484

\subsubsection{Совместное применение средств идентификации и аутентификации, встроенных в СУБД и в ОС}
СУБД Oracle предоставляет следующие способы аутентификации
пользователей:
\begin{enumerate}
    \item Аутентификация средствами операционной системы. Некоторые ОС
позволяют СУБД Oracle использовать информацию о пользователях, которыми
управляет собственно ОС. В этом случае пользователь компьютера имеет доступ к
ресурсам БД без дополнительного указания имени и пароля – используются его
сетевые учетные данные. Данный вид аутентификации считается небезопасным и
используется, в основном, для аутентификации администратора СУБД.
    \item Аутентификация при помощи сетевых сервисов. Данный вид аутентификации
обеспечивает опция сервера Oracle Advanced Security. Она предоставляет
возможность SSL-аутентификации, а так же аутентификацию с помощью служб
третьих сторон, в роли которых могут выступать Kerberos, PKI, RADIUS или служба
LDAP-каталога.
    \item Аутентификация в многоуровневых приложениях. Приведенные выше
методы аутентификации также могут быть применены и в многоуровневых
приложениях. Как правило, для доступа к приложениям из сети Интернет используется аутентификация по имени и паролю (в том числе с использованием протокола
RADIUS), либо по протоколу SSL. Прочие методы используются для работы
пользователей в локальной сети.
\end{enumerate}

Источник : http://www.iso27000.ru/chitalnyi
-zai/
zaschita-personalnyh-dannyh/
obespechenie-zaschity-personaln\%h-dannyh-v-subd-oracle
\subsection{Средства управления доступом}
\subsubsection{Основные понятия: субъекты и объекты, группы пользователей, привилегии, роли и представления.}
Сначала приведем формальные определения субьекта и обьекта доступа:

Субъект доступа – активная сущность АС (автоматизированной системы), которая может изменять состояние системы через порождение процессов над объектами, в том числе порождать новые объекты и инициализировать порождение новых субъектов.

Объект доступа – пассивная сущность АС, процессы над которой могут в определенных случаях быть источником порождения новых субъектов.
Теперь "по пацански" - чисто для понимания (хотя и неверно):
\begin{itemize}
    \item Объект – ресурс АС к которой обращается субьект (информация)
    \item Субъект – пользователь
\end{itemize}

Группа - это именованная совокупность пользователей

Роль – набор прав (полномочий, привилегий) или ролей

Привилегии – права доступа, предоставляемые субъекту (тут нужно прям уточнить в лоб - привилегия дается СУБЬЕКТУ)

Представления - это специфический образ таблицы или набора таблиц,
определенный оператором SELECT.В отличие от обычных таблиц реляционных баз данных, представление не является самостоятельной частью набора данных, хранящегося в базе. Содержимое представления динамически вычисляется на основании данных, находящихся в реальных таблицах. Изменение данных в реальной таблице базы данных немедленно отражается в содержимом всех представлений, построенных на основании этой таблицы. Некоторые СУБД имеют расширенные представления для данных, доступных только для чтения. Так, СУБД Oracle реализует концепцию «материализованных представлений» — представлений, содержащих предварительно выбранные невиртуальные наборы данных, совместно используемых в распределённых БД. Эти данные извлекаются из различных удалённых источников (с разных серверов распределённой СУБД). Целостность данных в материализованных представлениях поддерживается за счёт периодических синхронизаций или с использованием триггеров. Аналогичный механизм предусмотрен в Microsoft SQL Server версии 2000.

По самой сути представления могут быть доступны только для чтения. Тем не менее, в некоторых СУБД (например, в Oracle) представления могут быть редактируемыми, как и обычные физические таблицы.

\subsubsection{Виды привилегий: привилегии безопасности и доступа. Использование ролей и привилегий пользователей.}
Привилегии в СУБД можно подразделить на две категории: привилегии
безопасности и привилегии доступа. Привилегии безопасности позволяют
выполнять административные действия. Привилегии доступа, в соответствии с
названием, определяют права доступа субъектов к определенным объектам.

Привилегии безопасности всегда выделяются конкретному
пользователю (а не группе, роли или всем) во время его создания (оператором
CREATE USER) или изменения характеристик (оператором ALTER USER). Примеры
таких привилегий в СУБД INGRES:
\begin{itemize}
    \item security - право управлять безопасностью СУБД и отслеживать действия
пользователей. Пользователь с этой привилегией может подключаться к любой базе
данных, создавать, удалять и изменять характеристики пользователей, групп и ролей,
передавать права на доступ к базам данным другим пользователям, управлять
записью регистрационной информации, отслеживать запросы других пользователей.
    \item createdb - право на создание и удаление баз данных. Этой привилегией,
помимо администратора сервера, должны обладать пользователи, которым отводится
роль администраторов отдельных баз данных.
    \item operator - право на выполнение действий, которые традиционно относят к
компетенции оператора. Имеются в виду запуск и остановка сервера, сохранение и
восстановление информации. Помимо администраторов сервера и баз данных этой
привилегией целесообразно наделить также администратора операционной системы.
    \item maintain\_locations - право на управление расположением баз администратора
сервера баз данных и операционной системы.
\end{itemize}

Привилегии доступа выделяются пользователям, группам, ролям или всем посредством оператора GRANT и изымаются с помощью оператора REVOKE. Эти привилегии, как правило, присваивает владелец соответствующих объектов (он же - администратор базы данных) или обладатель привилегии security (обычно администратор сервера баз данных).

Прежде чем присваивать привилегии группам и ролям, их (группы и роли) необходимо создать с помощью операторов CREATE GROUP и CREATE ROLE.

Для изменения состава группы служит оператор ALTER GROUP. Оператор DROP GROUP позволяет удалять группы, правда, только после того, как опустошен список членов группы. Оператор ALTER ROLE служит для изменения паролей ролей, a DROP ROLE - для удаления ролей.

Создавать и удалять именованные носители привилегий, а также изменять их характеристики может лишь пользователь с привилегией security. При совершении подобных действий необходимо иметь подключение к базе данных, в которой хранятся сведения о субъектах и их привилегиях.
\subsubsection{Соотношение прав доступа СУБД и операционной системы}
Данные, хранящиеся средствами СУБД, располагаются в файлах и/или логических томах операционной системы. Соответственно, и доступ к этим данным возможен как с помощью СУБД, так и посредством утилит ОС. Так называемая естественная защита баз данных, являющаяся следствием относительно сложного формата их хранения, едва ли способна остановить злоумышленника.

Чтобы средствами ОС нельзя было скомпрометировать базу данных и сопутствующую информацию, например журналы транзакций, необходимо установить максимально жесткий режим доступа к соответствующим файлам и томам. Для UNIX-систем целесообразно предоставить право непосредственного доступа только пользователям root, ingres и, быть может, администраторам баз данных. Все прочие субъекты должны осуществлять доступ к базам с помощью программ со взведенным битом переустановки действующего идентификатора пользователя.
Данные из базы могут экспортироваться в файлы операционной системы или другие хранилища. Возможен и обратный процесс импорта данных. Необходимо следить за тем, чтобы подобные операции не понижали уровня защищенности информации. Сделать это, вообще говоря, непросто. Во-первых, операционная система (например, персонального компьютера) может не обеспечивать должной защиты. Во-вторых, даже для развитых ОС необходимо установить и поддерживать соответствие между механизмами защиты СУБД и операционных систем. При большом числе пользователей (порядка нескольких сотен) данная задача становится очень сложной. Видимо, наиболее практичное решение сводится к административному контролю за экспортом/импортом информации.
Можно предположить, что исходные тексты сколько-нибудь сложных процедур баз данных будут храниться в файлах операционной системы, формально не имеющих отношения к СУБД. Потенциально возможно нелегальное изменение исходного текста, которое, как было показано ранее, способно привести к серьезным нарушениям информационной безопасности. Вероятно, и здесь лучшим средством является административный контроль за размещением процедур в базах данных и за передачей прав на их выполнение.
СУБД и ОС предлагают во многом сходные средства защиты данных. Более того, многие СУБД просто полагаются на операционную систему в плане идентификации и проверки подлинности пользователей. Тем не менее даже для таких "дружественных" сервисов проведение в жизнь согласованной политики безопасности является очень сложным делом. На практике приходится всячески ограничивать информационный обмен между СУБД и ОС.
\subsubsection{ Метки безопасности.}


В "Критериях оценки надежных компьютерных систем" применительно к системам уровня безопасности описан механизм меток безопасности, реализованный в версии INGRES/ Enhanced Security (INGRES с повышенной безопасностью). Применять эту версию на практике имеет смысл только в сочетании с операционной системой и другими программными компонентами того же уровня безопасности. Тем не менее рассмотрение реализации меточной безопасности в СУБД INGRES интересно с познавательной точки зрения, а сам подход, основанный на разделении данных по уровням секретности и категориям доступа, может оказаться полезным при проектировании системы привилегий многочисленных пользователей по отношению к большим массивам данных.

В СУБД INGRES/Enhanced Security к каждой реляционной таблице неявно добавляется столбец, содержащий метки безопасности строк таблицы. Метка безопасности состоит из трех компонентов:
· уровень секретности. Смысл этого компонента зависит от приложения. В частности, возможен традиционный спектр уровней от "совершенно секретно" до "несекретно";
· категории. Понятие категории позволяет разделить данные на "отсеки" и тем самым повысить надежность системы безопасности. В коммерческих приложениях категориями могут служить "финансы", "кадры", "материальные ценности" и т.п. Ниже назначение категорий разъясняется более подробно;
· области. Является дополнительным средством деления информации на отсеки. На практике компонент "область" может действительно иметь географический смысл, обозначая, например, страну, к которой относятся данные.
Каждый пользователь СУБД INGRES/ Enhanced Security характеризуется степенью благонадежности, которая также определяется меткой безопасности, присвоенной данному пользователю. Пользователь может получить доступ к данным, если степень его благонадежности удовлетворяет требованиям соответствующей метки безопасности. Более точно:
· уровень секретности пользователя должен быть не ниже уровня секретности данных;
· набор категорий, заданных в метке безопасности данных, должен целиком содержаться в метке безопасности пользователя;
· набор областей, заданных в метке безопасности пользователя, должен целиком содержаться в метке безопасности данных.

Рассмотрим ГИПОТЕТИЧЕСКИЙ пример И НИКОГО НЕ ИМЕЕМ ВВИДУ, ВСЕ ПЕРСОНАЖИ И СОБЫТИЯ ЯВЛЯЮТСЯ ФАНТАЗИЕЙ БОЛЬНОЙ ГОЛОВЫ АВТОРА. Пусть данные о том, ехал человек в поезде или нет имеют уровень секретности "для служебного пользования", принадлежат категории "поезда" и "пассажиры" и относятся к областям "Москва" и "Татарстан". Далее, пусть степень благонадежности пользователя характеризуется меткой безопасности с уровнем секретности "для служебного пользования", категориями "поезда" и "самолеты", а также областью "Москва" и "Татарстан". Такой пользователь не получит доступ к данным, в метке пользователя была указана только категория "поезда" (без категории "пассажиры"), в доступе к данным ему было бы отказано.

Когда пользователь производит выборку данных из таблицы, он получает только те строки, меткам безопасности которых удовлетворяет степень его благонадежности. Для совместимости с обычными версиями СУБД, столбец с метками безопасности не включается в результирующую информацию.

Отметим, что механизм меток безопасности не отменяет, а дополняет произвольное управление доступом. Пользователи по-прежнему могут оперировать таблицами только в рамках своих привилегий, но даже при наличии привилегии SELECT им доступна, вообще говоря, только часть данных.

При добавлении или изменении строк они, как правило, наследуют метки безопасности пользователя, инициировавшего операцию. Таким образом, даже если авторизованный пользователь перепишет секретную информацию в общедоступную таблицу, менее благонадежные пользователи не смогут ее прочитать.

Специальная привилегия, DOWNGRADE, позволяет изменять метки безопасности, ассоциированные с данными. Подобная возможность необходима, например, для коррекции меток, по тем или иным причинам оказавшихся неправильными.

Представляется естественным, что СУБД INGRES/Enhanced Security допускает не только скрытое, но и явное включение меток безопасности в реляционные таблицы. Появился новый тип данных, security label, поддерживающий соответствующие операции сравнения.

\subsubsection{Использование представлений для обеспечения конфиденциальности
информации в СУБД.}

СУБД предоставляют специфическое средство управления доступом - представления. Представления позволяют сделать видимыми для субъектов определенные столбцы базовых таблиц (реализовать проекцию) или отобрать определенные строки (реализовать селекцию). Не предоставляя субъектам прав доступа к базовым таблицам и сконструировав подходящие представления, администратор базы данных защитит таблицы от несанкционированного доступа и снабдит каждого пользователя своим видением базы данных, когда недоступные объекты как бы не существуют.

Приведем пример создания представления, содержащего два столбца исходной таблицы и включающего в себя только строки с определенным значением одного из столбцов:
\begin{lstlisting}[style=mystyle]
CREATE VIEW empview AS
SELECT name, dept
FROM employee
WHERE dept = "shoe";
\end{lstlisting}
Предоставим всем право на выборку из этого представления:
\begin{lstlisting}[style=mystyle]
GRANT SELECT
ON empview
TO PUBLIC;
\end{lstlisting}
Субъекты, осуществляющие доступ к представлению empview, могут пытаться запросить сведения об отделах, отличных от shoe, например:
\begin{lstlisting}[style=mystyle]
SELECT *
FROM empview
WHERE dept = "toy";
\end{lstlisting}
но в ответ просто получат результат из нуля строк, а не код ответа, свидетельствующий о нарушении прав доступа. Это принципиально важно, так как лишает злоумышленника возможности получить список отделов косвенным образом, анализируя коды ответов, возвращаемые после обработки SQL-запросов.
\subsection{Обеспечение конфиденциальности путем тиражирования БД}

Тиражирование (репликация) – технология, предусматривающая поддержку
копий всей БД или ее фрагментов в нескольких узлах сети. Копия БД называется
репликой. Копии БД обычно приближены к местам использования информации.
В контексте информационной безопасности тиражирование можно рассматривать как средство повышения доступности данных. Стала легендой история про бакалейщика из Сан-Франциско, который после разрушительного землетрясения восстановил свою базу данных за 16 минут, перекачав из другого города предварительно протиражированную информацию.

Развитые возможности тиражирования предоставляет СУБД INGRES. Им посвящена статья [2]. Здесь мы рассмотрим возможности другого популярного сервера СУБД - Informix OnLine Dynamic Server (OnLine-DS) 7.1. В отличие от предыдущего раздела, речь пойдет об обычных (а не кластерных) конфигурациях.
В Informix OnLine-DS 7.1 поддерживается модель тиражирования, состоящая в полном отображении данных с основного сервера на вторичные.

В конфигурации серверов Informix OnLine-DS с тиражированием выделяется один основной и ряд вторичных серверов. На основном сервере выполняется и чтение, и обновление данных, а все изменения передаются на вторичные серверы, доступные только на чтение. В случае отказа основного сервера вторичный автоматически или вручную переводится в режим доступа на чтение и запись. Прозрачное перенаправление клиентов при отказе основного сервера не поддерживается, но оно может быть реализовано в рамках приложений.

После восстановления основного сервера возможен сценарий, при котором этот сервер становится вторичным, а бывшему вторичному, который уже функционирует в режиме чтения-записи, придается статус основного; клиенты, которые подключены к нему, продолжают работу. Таким образом, обеспечивается непрерывная доступность данных.
Тиражирование осуществляется путем передачи информации из журнала транзакций (логического журнала) в буфер тиражирования основного сервера, откуда она пересылается в буфер тиражирования вторичного сервера. Такая пересылка может происходить либо в синхронном, либо в асинхронном режиме. Синхронный режим гарантирует полную согласованность баз данных - ни одна транзакция, зафиксированная на основном сервере, не останется незафиксированной на вторичном, даже в случае сбоя основного сервера. Асинхронный режим не обеспечивает абсолютной согласованности, но улучшает рабочие характеристики системы.

Побочный положительный эффект тиражирования - возможность вынести преимущественно на вторичный сервер ресурсоемкие приложения поддержки принятия решений. В этом случае они могут выполняться с максимальным использованием средств параллельной обработки, не подавляя приложений оперативной обработки транзакций, сосредоточенных на основном сервере. Это также можно рассматривать как фактор повышения доступности данных.

Формальная модель для обеспечения конфиденциальности БД с помощью тиражирования.
Вот не нашел источников под это.
Архитектура и политика безопасности в модели SINTRA.

https://pdfs.semanticscholar.o
rg/6065/
91924dfdd
a7ac91300a
b9ca9aa7070bc5a5c.pdf?
\_ga=2.24881394.95010953.1590951714-1165528036.1590951714 глава 4 на англуйцком было слажна и предлагается читателю перевести это самому
\subsection{Аудит и подотчетность}
Регистрация действий пользователей, так или иначе влияющих на информационную безопасность, является фактором, сдерживающим потенциальных злоумышленников и позволяющим расследовать уже случившиеся нарушения.

Более точно, журнал регистрационной информации может использоваться для следующих целей:
\begin{itemize}
    \item обнаружение необычных или подозрительных действий пользователей и идентификации лиц, совершающих эти действия;
    \item обнаружение попыток несанкционированного доступа
    \item оценка возможных последствий состоявшегося нарушения информационной безопасности
    \item оказание помощи в расследовании случаев нарушения безопасности
    \item организация пассивной защиты от нелегальных действий. Пользователи, зная, что их действия фиксируются, могут не решиться на незаконные операции. Чтобы данная цель достигалась, необходимо довести до сведения каждого пользователя, что каждое его действие регистрируется и что за незаконные операции он может понести наказание.
\end{itemize}



\subsubsection{Подотчетность действий пользователя и аудит связанных с безопасностью событий.}
Протоколирование, или регистрация, представляет собой механизм подотчётности системы обеспечения информационной безопасности, фиксирующий все события, относящиеся к вопросам безопасности. В свою очередь, аудит – это анализ протоколируемой информации с целью оперативного выявления и предотвращения нарушений режима информационной безопасности.

Не менее важен вопрос о порядке хранения системных журналов. Поскольку файлы журналов хранятся на том или ином носителе, неизбежно возникает проблема переполнения максимально допустимого объёма системного журнала. При этом реакция системы может быть различной, например:
\begin{itemize}
    \item система может быть заблокирована вплоть до решения проблемы с доступным дисковым пространством
    \item могут быть автоматически удалены самые старые записи системных журналов
    \item система может продолжить функционирование, временно приостановив протоколирование информации
\end{itemize}

Безусловно, последний вариант в большинстве случаев является неприемлемым, и порядок хранения системных журналов должен быть чётко регламентирован в политике безопасности организации.

В терминологии регистрационной службы любое действие, способное изменить состояние базы данных, называется событием и может регистрироваться. В СУБД Informix события обозначаются четырехбуквенными мнемониками. Приведем несколько событий безопасности:
\begin{itemize}
    \item ACTB - доступ к таблице
    \item CLDB - закрытие таблицы
    \item DRTB - удаление таблицы
\end{itemize}


\subsubsection{Регистрация действий пользователя}
Регистрация всех событий для всех пользователей существенно снизит эффективность работы СУБД. Администратор СУБД (или лицо, отвечающее за информационную безопасность) должен выбрать приемлемый баланс между безопасностью и эффективностью.
Рекомендуется регистрировать по крайней мере следующие события:
\begin{itemize}
    \item передача привилегий доступа к базе данных (GRDB)
    \item лишение привилегий доступа к базе данных (RVDB)
    \item передача привилегий доступа к таблице (GRTB)
    \item лишение привилегий доступа к таблице (RVTB)
    \item открытие базы данных (OPDB)
\end{itemize}

Перечисленные события происходят нечасто, однако их фиксация позволяет составить представление о том, чем интересуется каждый из пользователей. Если какой-либо пользователь замечен в подозрительных действиях, для него можно включить режим регистрации всех событий. Для администратора СУБД и лица, отвечающего за безопасность, подобный режим должен быть включен постоянно.
Отметим, что у пользователя нет возможности узнать, какие из его действий регистрируются. Целесообразно поддерживать в нем уверенность, что регистрируется все или почти все.

\subsubsection{Управление набором регистрируемых событий}

Для управления набором регистрируемых событий в СУБД Informix используются маски событий. Три стандартные маски с именами \_default, \_require и \_exclude формируют стандартное регистрационное окружение. Кроме того, для отдельных пользователей могут быть заданы персональные маски с именами, совпадающими с входными именами этих пользователей.

Результирующее регистрационное окружение пользователя формируется следующим образом:
Берется маска \_default или маска пользователя, если таковая имеется.

В число регистрируемых дополнительно включаются события, заданные маской \_require.
Из числа регистрируемых исключаются события, заданные маской \_exclude, при условии, что они не были упомянуты в маске \_require.

Таким образом, в СУБД Informix выполнено одно из требований к системам класса безопасности C2, предписывающее возможность задания своего перечня регистрируемых событий для каждого пользователя.

Маски можно формировать с помощью утилиты onaudit. Приведем пример командной строки, позволяющей добавить к маске \_default регистрацию событий, вызываемых операциями со строками таблиц:
onaudit -m -u \_default -e +DLRW,INRW,RDRW,UPRW

С помощью утилиты onaudit можно выдать состояние той или иной маски, создать, модифицировать или удалить ее. Та же утилита (и это важно отметить) позволяет включать и выключать регистрацию событий сервером СУБД Informix.

\subsubsection{Анализ регистрационной информации}
\begin{itemize}
    \item ONLN - фиксированное поле, обозначающее события, фиксируемые сервером Informix-OnLine
    \item дата и время события
    \item имя клиентского компьютера, инициировавшего событие
    \item идентификатор клиентского процесса, инициировавшего событие
    \item имя серверного компьютера
    \item имя пользователя
    \item код завершения действия, вызвавшего событие
    \item мнемоника события
    \item дополнительные поля, идентифицирующие базы данных, таблицы и другие объекты, вовлеченные в событие
\end{itemize}

С помощью утилиты onshowaudit можно отобрать часть регистрационной информации и организовать ее просмотр или, воспользовавшись утилитой dbload, загрузить ее в базу данных и анализировать затем SQL-средствами. Регистрационная служба СУБД Informix является субъектно-ориентированной в том смысле, что можно задать свой набор отслеживаемых событий для каждого пользователя (субъекта), однако нет возможности указать имена объектов (таблиц, процедур и т.п.), операции с которыми отслеживались бы особым образом. Вместо этого предлагается полагаться на средства анализа регистрационной информации. Очевидно, после загрузки регистрационного журнала в базу данных можно отобрать сведения по сколь угодно сложному критерию и сгенерировать отчет, по существу, произвольного вида.

Регистрационные файлы и результат их загрузки в базу данных нуждаются в защите. В частности, для файлов рекомендуется установить в качестве владельца пользователя root, владеющей группой сделать informix, а режим доступа положить равным 0660 (доступ на чтение и запись только для владельца и группы).

Регистрационная информация нуждается в ежедневном анализе. В противном случае реакция на подозрительные действия или нарушения окажется запоздалой. В первом приближении подозрительными можно считать действия, завершившиеся ненормальным образом, то есть с ненулевым кодом. Более сложные эвристики могут опираться на статистический анализ спектра пользовательских действий. Как уже указывалось, после обнаружения подозрительной активности целесообразно включить режим регистрации всех действий пользователя и/или принять административные меры.

Источник на котором основана большая часть данной главы:

https://www.osp.ru/news
/articles/1996/0131/13031467
