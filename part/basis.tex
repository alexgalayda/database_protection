\section{Теоретические основы безопасности в СУБД}

\subsection{Критерии защищенности БД}

\paragraph{Критерии оценки надежных компьютерных систем (TCSEC)}
\paragraph{Понятие политики безопасности}
\paragraph{Совместное применение различных политик безопасности в рамках единой модели}
\paragraph{Интерпретация TCSEC для надежных СУБД (TDI)}
\paragraph{Оценка надежности СУБД как компоненты вычислительной системы}
\paragraph{Монитор ссылок}
\paragraph{Применение TCSEC к СУБД непосредственно}
\paragraph{Элементы СУБД, к которым применяются TDI: метки, аудит, архитектура системы, спецификация, верификация, проектная документация}
\paragraph{Критерии безопасности ГТК}

\subsection{Модели безопасности в СУБД}

\paragraph{Дискреционная (избирательная) и мандатная (полномочная) модели безопасности}
\paragraph{Классификация моделей}
\paragraph{Аспекты исследования моделей безопасности}
\paragraph{Особенности применения моделей безопасности в СУБД}
\paragraph{Дискреционные модели: HRU, Take-Grant, Action-Entity, Wood}
\paragraph{Мандатные модели: Bell-LaPadula, Biba, Dion, Sea View, Jajodia\&Sandhu, Smith\&Winslett, решеточная}
\paragraph{БД с многоуровневой секретностью (MLS)}
\paragraph{Многозначность}