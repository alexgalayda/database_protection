\section{Механизмы обеспечения целостности СУБД}

\subsection{Угрозы целостности СУБД}

\paragraph{Основные виды и причины возникновения угроз целостности}
\paragraph{Способы противодействия}

\subsection{Метаданные и словарь данных}

\paragraph{Назначение словаря данных}
\paragraph{Доступ к словарю данных}
\paragraph{Состав словаря}
\paragraph{Представления словаря}

\subsection{Понятие транзакции}

\paragraph{Фиксация транзакции}
\paragraph{Прокрутки вперед и назад}
\paragraph{Контрольная точка}
\paragraph{Откат}
\paragraph{Транзакции как средство изолированности пользователей}
\paragraph{Сериализация транзакций}
\paragraph{Методы сериализации транзакций}

\subsection{Блокировки}

\paragraph{Режимы блокировок}
\paragraph{Правила согласования блокировок}
\paragraph{Двухфазный протокол синхронизационных блокировок}
\paragraph{Тупиковые ситуации, их распознавание и разрушение}

\subsection{Ссылочная целостность}

\paragraph{Декларативная и процедурная ссылочные целостности}
\paragraph{Внешний ключ}
\paragraph{Способы поддержания ссылочной целостности}

\subsection{Правила(триггеры)}
\paragraph{Цели использования правил}
\paragraph{Способы задания, моменты выполнения}

\subsection{События}

\paragraph{Назначение механизма событий}
\paragraph{Сигнализаторы событий}
\paragraph{Типы уведомлений о происхождении события}
\paragraph{Компоненты механизма событий}